\section{Conclusiones}
En este experimento pudimos apreciar una aplicación de la teoría de la información para poder descubrir partes de una red cuyo impacto es más destacado en la misma, en particular, nodos y protocolos con mayor importancia.
\\
En la primer parte del trabajo, al buscar protocolos diferenciados, se descubrió que el protocolo más usado es el de IPV4, lo que es razonable. Además se notó que en algunas redes el protocolo IPV6 es bastante frecuente y que el protocolo ARP siempre se encontró presente en ellas (esto último es previsible, por el funcionamiento de las redes IP en LAN)
\\
En la segunda parte del trabajo se investigó la existencia de nodos distinguidos (o símbolo distinguido en el contexto de la fuente $S_1$). Se detectó que en cada una de las redes el router fue uno de los nodos distinguidos. También se observó que en las redes no controladas la entropía de la fuente $S_1$ es menor que en aquellas que sí son controladas, esto se puede deber a que la entropía mide la previsibilidad de la fuente, por lo que en redes no controladas éstas será mayor.

