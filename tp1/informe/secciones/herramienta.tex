\section{Herramienta desarrollada}

Para realizar las capturas de las redes, se desarrolló un programa en lenguaje Python integrado con la herramienta Scapy.
Este programa captura el tráfico de la red en modo promiscuo. La información de los paquetes leídos es almacenada en diferentes archivos para su posterior análisis.

En estos archivos se almacenan datos como, la cantidad de paquetes, la información, la probabilidad y la entropía. Estos datos están relacionados con las dos fuentes elegidas: por tipo de protocolo para los paquetes Ethernet y por IP para la captura de paquetes de tipo ARP.

Además, parte de estos datos también se almacenan en un archivo con formato JSON. Este archivo es utilizado por otro programa también desarrollado en Python para generar gráficos y así poder analizar la información capturada de la red.
\\

Estos programas se pueden ejecutar desde una terminal con sistema operativo Linux u OS X, con lenguaje Python y las librerías Scapy y Matplotlib.

Para la ejecución de la herramienta de captura de paquetes de la red, se debe ejecutar el siguiente comando:

\begin{lstlisting}[language=bash]
  $ sudo ./sniffer.py <file_preffix> <timeout> 
\end{lstlisting}

Para la ejecución del programa que genera los gráficos, se debe ejecutar:

\begin{lstlisting}[language=bash]
  $ sudo ./plot.py <file_preffix> 
\end{lstlisting}

Donde:

\begin{itemize}
\item file\_preffix: Es el prefijo que se le agregará al nombre del archivo JSON generado por el programa sniffer.py y el prefijo con el cual plot.py lo ubicará. Además es el prefijo que plot.py le colocará a los nombres de los archivos de los gráficos generados.
\item timeout: Es el tiempo durante el cual el programa se encontrará capturando tráfico de la red.
\end{itemize}