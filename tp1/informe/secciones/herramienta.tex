\section{Primera consigna: capturando tráfico}

\subsection{Herramienta}

Para realizar las capturas de las redes, se desarrollo un script en el lenguaje Python integrado con la herramienta Scapy.
Este script escucha pasivamente la red capturando los paquetes que son almacenados en distintos archivos para su posterior análisis.

En los archivos se guarda la información, probabilidad y entropía de las 2 fuentes elegidas. Por tipo de protocolo para los paquetes Ethernet y por IP para la captura de paquetes ARP.

Un 3 archivo de formato json, contiene la información de manera organizada para poder ser leído por otro script desarrollado utilizado para generar distintos gráficos.

Este segundo script genera gráficos como histogramas, de torta y grafos según paquetes ARP enviados.

\subsection{Ejecución}

Para la ejecución de la herramienta, debemos ejecutar los siguiente comandos desde una terminal en el sistema operativo Linux.

\begin{lstlisting}[language=bash]
  $ sudo ./sniffer.py <outputfile> <timeout> 
\end{lstlisting}

Para la ejecución del graficador:

\begin{lstlisting}[language=bash]
  $ sudo ./plot.py <outputfile> 
\end{lstlisting}
