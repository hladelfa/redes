\section{Resúmen}

El objetivo de este trabajo es el de experimentar con las herramientas provistas por el protocolo ICMP para analizar las rutas tomadas por los paquetes hasta alcanzar su destino. Además se analizará la existencia de saltos entre nodos intercontinentales en las rutas de los experimentos.

\section{Introducción}

Para este trabajo implementamos nuestra propia versión de la herramienta traceroute, en lenguaje Python y utilizando la librería Scapy. El funcionamiento de nuestro programa es similar al del traceroute que se encuentra en los sistemas operativos más conocidos. Para lograr encontrar la ruta que podrían seguir los paquetes para alcanzar a un host destino lo que hacemos es ir enviando paquetes ICMP Echo Request, comenzando con un valor de TTL igual a uno e incrementando este valor gradualmente. De esta forma, cada router al recibir estos paquetes decrementará el valor del TTL en uno. Si, luego de decrementar el valor del TTL, el mismo queda en cero, el router que posee el paquete responderá al host origen con un mensaje ICMP Time Exceeded. Cuando un paquete alcanza finalmente al host destino, este responderá con un mensaje ICMP Echo Reply. De esta forma, recibiendo los sucesivos mensajes ICMP Time Exceeded de cada router y con el mensaje final ICMP Echo Reply, podremos armar una ruta con los routers intermedios hasta el host destino.

Para utilizar nuestra versión de traceroute se debe ejecutar:

\begin{lstlisting}[language=bash]
  $ python traceroute.py <host> 
\end{lstlisting}

Donde:

\begin{itemize}
\item host: Es la dirección IP o nombre de dominio del host destino hasta el que se quiere calcular la ruta.
\end{itemize}