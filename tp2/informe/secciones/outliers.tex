\section{Estimación de outliers}

Basándonos en la técnica de estimación propuesta por John M. Cimbala para encontrar los outliers de una muestra, agregamos a nuestro programa los cálculos necesarios para inferir saltos intercontinentales en las rutas tomadas por los paquetes hacia un host destino.
Los pasos que tomamos para realizar estos cálculos fueron:

Para cada TTL de los ICMP Echo Request enviados, medimos el RTT de cada salto entre cada par de nodos. Utilizamos al conjunto de los $RTT_i$ como los valores de nuestra muestra, con i=1,…,n y donde n es la cantidad de muestras obtenidas.
Por cada $RTT_i$ calculamos el valor absoluto del desvío como:
\[
\delta_i = |RTT_i - \overline{RTT}|
\]

Donde $\overline{RTT}$ es el valor de la media de la muestra calculado como el promedio de los $RTT_i$ medidos.

Tomamos como referencia la tabla de valores calculados para la fórmula de Thompson modificada del artículo de Cimbala para obtener el valor  $\tau$ correspondiente a las n muestras. Con este valor obtenemos:
\[
\tau S = \tau * S
\]

Donde S es el desvío estándar calculado como
\[
S=\sqrt{\frac{\sum_{i=1}^{n} (RTT_i - \overline{RTT})^2}{n-1}}
\]

Luego verificamos, si $\delta_i > \tau S$, entonces asumimos que el salto con $RTT_i$ es un enlace intercontinental.