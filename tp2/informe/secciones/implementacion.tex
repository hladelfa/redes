\section{Implementación}

Para este trabajo implementamos nuestra propia versión de la herramienta \texttt{traceroute}, en lenguaje \emph{Python} y utilizando la librería \emph{Scapy}. El funcionamiento de nuestro programa es similar al del \texttt{traceroute} que se encuentra en los sistemas operativos más conocidos. Para lograr encontrar la ruta que podrían seguir los paquetes para alcanzar a un host destino lo que hacemos es ir enviando paquetes \emph{ICMP Echo Request}, comenzando con un valor de TTL igual a uno e incrementando este valor gradualmente. De esta forma, cada router al recibir estos paquetes decrementará el valor del TTL en uno. Si, luego de decrementar el valor del TTL, el mismo queda en cero, el router que posee el paquete responderá al host origen con un mensaje \emph{ICMP Time Exceeded}. Cuando un paquete alcanza finalmente al host destino, este responderá con un mensaje \emph{ICMP Echo Reply}. De esta forma, recibiendo los sucesivos mensajes \emph{ICMP Time Exceeded} de cada router y con el mensaje final \emph{ICMP Echo Reply}, podremos armar una ruta con los routers intermedios hasta el host destino.
Para evitar que el programa nunca finalize por intentar alcanzar al destino cuando este no responde, agregamos una cota al valor del TTL de 40.
También, para cada valor de TTL realizamos hasta 10 intentos para encontrar la ruta. De esta forma podría darse la situación en que encontremos más de una ruta posible. Por lo tanto, si obtenemos al menos 3 veces consecutivas respuesta del mismo nodo, lo tomamos como parte de la ruta que retornaremos como respuesta y su RTT será el promedio de sus muestras.

Además, a este programa le agregamos los cálculos necesarios para encontrar los enlaces entre nodos cuyos valores de RTT (Roundtrip Time) se encuentran considerablemente por encima del resto. Es decir, que buscaremos encontrar los outliers de las muestras. Estos cálculos los utilizaremos para inferir cuáles son los enlaces intercontinentales. Para encontrar a estos outliers utilizaremos la técnica de estimación de John M. Cimbala propuesta por la cátedra. Veremos más detalles sobre estos cálculos en la siguiente sección.

Para utilizar nuestra versión de \texttt{traceroute} se debe ejecutar:

\begin{lstlisting}[language=bash]
  $ python traceroute.py <host> 
\end{lstlisting}

Donde:

\begin{itemize}
\item host: Es la dirección IP o nombre de dominio del host destino hasta el cual se quiere calcular la ruta.
\end{itemize}