\section{Implementación}

Para este trabajo implementamos nuestra propia versión de la herramienta \texttt{traceroute}, en lenguaje \emph{Python} y utilizando la librería \emph{Scapy}.
\\
El funcionamiento general de nuestro programa es similar al del \texttt{traceroute} que se encuentra en los sistemas operativos más conocidos. Para lograr encontrar la ruta que podrían seguir los paquetes hasta alcanzar a un host destino lo que hacemos es ir enviando paquetes \emph{ICMP Echo Request}, comenzando con un valor de TTL (Time To Live) igual a uno e incrementando este valor gradualmente. De esta forma, cada router al recibir estos paquetes decrementará el valor del TTL del paquete en uno. Si, luego de decrementar el valor del TTL, el mismo queda en cero, el router que posee el paquete responderá al host origen con un mensaje \emph{ICMP Time Exceeded}. Cuando un paquete alcanza finalmente al host destino, este responderá con un mensaje \emph{ICMP Echo Reply}. De esta forma, recibiendo los sucesivos mensajes \emph{ICMP Time Exceeded} de cada router y con el mensaje final \emph{ICMP Echo Reply}, podremos armar una ruta con los routers intermedios hasta el host destino.
\\\\
Para evitar que el programa nunca finalice por intentar alcanzar al destino cuando este no responde, agregamos una cota al valor del TTL de 40.
Además, para cada valor de TTL realizamos hasta 30 intentos para encontrar el correspondiente nodo y nos quedaremos con el promedio del valor de los RTT (Roundtrip Time). De esta forma podría darse la situación de que encontremos más de un nodo posible. En este caso tomaremos al que se presentó con mayor frecuencia y como RTT al promedio de los RTT obtenidos.
\\\\
Además, a este programa le agregamos los cálculos necesarios para encontrar los enlaces entre nodos cuyos valores de RTT se encuentran considerablemente por encima del resto. Es decir, que buscaremos encontrar los outliers de las muestras. Estos cálculos los utilizaremos para inferir cuáles son los enlaces intercontinentales. Para encontrar a estos outliers utilizaremos la técnica de estimación de John M. Cimbala propuesta por la cátedra. Veremos más detalles sobre estos cálculos en la siguiente sección.
\\\\
Para utilizar nuestra versión de \texttt{traceroute} se debe ejecutar:

\begin{lstlisting}[language=bash]
  $ python traceroute.py <host> <prefijo>
\end{lstlisting}

Donde:

\begin{itemize}
\item host: Es la dirección IP o nombre de dominio del host destino hasta el cual se quiere calcular la ruta.
\item prefijo: Es el prefijo que se utilizará para generar los nombres de los archivos de salida.
\end{itemize}